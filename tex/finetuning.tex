\chapter{Fine-Tuning Techniques}

Full fine-tuning, of course, has the best performance but is computationally expensive. Need parameter-efficient fine-tuning (PEFT).

\begin{itemize}
    \item Adapter-based methods
    \begin{itemize}
        \item Adapters: small, trainable models inserted into pre-trained transformers.
        \item Freeze original model weights.
    \end{itemize}
    \item Instruction tuning
\end{itemize}

\section{Low-Rank Adaptation (LoRA)}
Introduced in \cite{lora}.
\begin{itemize}
    \item Intuition: only train low rank perturbations of the (selected) weight matrices.
    \item Let $W_0 \in \R^{d \times k}$ be a pre-trained weight matrix.
    \item Update: $W_0 + \Delta W = W_0 + BA,$ where $B \in \R^{d \times r}, A \in \R^{r \times d}, r \ll \min(d,k).$
    \item Advantages:
    \begin{itemize}
        \item Roughly converges to full fine-tuning as $r$ increases.
        \item No additional inference latency: when needing to switch to another downstream task, can recover $W_0$ by subtracting $BA,$ and then we can add a new $B'A'.$
    \end{itemize}
    \item Notes \cite{lora_notes}:
    \begin{itemize}
        \item Optimal placement highly dependent on the dataset and model architecture.
        \item For transformers, applying LoRA exclusively to attention layers provides the most stability and mitigates the risk of divergence.
        \item For MoE, applying LoRA to each expert individually boosts performance, but significantly increases memory usage. Applying to router gives limited success.
        \item Could optimize memory by using same $B$ across different $A.$
    \end{itemize}
\end{itemize}

\section{QLoRA}
Quantized LoRA, introduced in \cite{qlora}.

\subsection{4-bit NormalFloat Quantization}

\begin{itemize}
    \item Better quantization data type for normally distributed data.
    \item In general, for $k$-bit NormalFloat, equally divide the normal distribution into $2^k$ quantiles.
    \item Note: to ensure $0$ has an exact representation, actually divide the negative part into $2^{k-1}$ quantiles and the positive part into $2^{k-1} + 1$ quantiles, then remove one of the two overlapping zeros.
    \item Then
    \begin{equation*}
        X^{\nf 4} = \round\left(\frac{1}{\absmax(X^{\fp 32})} X^{\fp 32}\right) = \round(c^{\fp 32}X^{\fp 32}),
    \end{equation*}
    where $c^{\fp 32}$ is the \textit{quantization constant}.
\end{itemize}

\subsection{Block-wise Quantization}
\begin{itemize}
    \item One problem is quantization is that outliers severely impacts the scaling, and the full range of the lower precision data type may not be effectively used.
    \item Solution: Chunk the input tensor into blocks that are independently quantized, each with its own quantization constant.
\end{itemize}

\subsection{Double Quantization}
\begin{itemize}
    \item Helps reduce the memory footprint of quantization constants.
    \item Block-wise $k$-bit quantization for the quantization constants $c^{\fp 32}.$
    \item Gives $c^{\kbit}_2,$ with second level of quantization constants $c_1^{\fp 32}.$
\end{itemize}

\subsection{Implementation}
\begin{itemize}
    \item Recall: LoRA has
    \begin{equation*}
        Y = XW + XL_1L_2
    \end{equation*}
    for low rank matrices $L_i.$
    \item For QLoRA, do
    \begin{equation*}
        Y^{\bfl 16} = X^{\bfl 16}\doubleDequant(c_1^{\fp 32}, c_2^{\kbit}, W^{\nf 4}) + X^{\bfl 16}L_1^{\bfl 16}L_2^{\bfl 16},
    \end{equation*}
    where
    \begin{equation*}
        \doubleDequant(c_1^{\fp 32}, c_2^{\kbit}, W^{\nf 4}) = \dequant(\dequant(c_1^{\fp 32}, c_2^{\kbit}), W^{\nf 4}) = W^{\bfl 16}
    \end{equation*}
    \item Original paper \cite{qlora} uses $\fp 8$ for $c_2,$ block size of 64 for $W,$ and block size of 256 for $c_2.$
    \item Computations done in $\bfl 16$ and storage of $W$ in $\nf 4.$
\end{itemize}

\subsection{\texorpdfstring{\iat}{IA3}}
\begin{itemize}
    \item Introduced in \cite{ia3}, Infused Adapter by Inhibiting and Amplifying Inner Activations is intended to improve over LoRA.
    \item Notation: if $l \in \R^d$ and $A \in \R^{m \times d},$ then $(l \odot A)_{ij} = l_jA_{ij},$ i.e., element-wise multiplication of $l$ and rows of $A.$
    \item Learned vectors $l_v,l_k,l_f.$
    \item Attention layer:
    \begin{equation*}
        \softmax\left(\frac{Q(l_k \odot K^T)}{\sqrt{d_k}}\right) \times (l_v \odot V)
    \end{equation*}
    \item Feedforward layer:
    \begin{equation*}
        (l_f \odot \gamma(W_1x))W_2,
    \end{equation*}
    where $\gamma$ is the FFN nonlinearity.
\end{itemize}