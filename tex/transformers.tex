\chapter{Transformers}

\begin{itemize}
\item 
Encoding component is a stack (six in original paper) of encoders with same structure but different weights. Decoding component is a stack of decoders of the same number.

\item 
Encoders: Self-Attention \textrightarrow{} FNN
	\begin{itemize}
	\item 
	First encoder receives embedding of words.
	
	\item 
	Other encoders receive output of encoder directly below.
	\end{itemize}

\item 
Decoders: Self-Attention \textrightarrow{} Encoder-Decoder Attention \textrightarrow{} FNN
\end{itemize}

\section{Ingredients}
\subsection{Self-attention}
Process:
\begin{enumerate}
\item Create three vectors from each of the encoder’s input vectors: Query, Key, and Value vectors.

\item Note: in original paper, input vectors are in $\R^{512},$ and QKV vectors are in $\R^{64}.$

\item Self-attention score between $i$-th and $j$-th word is $q_i \cdot k_j$ for all $j.$

\item Divide scores by square root of dimension (8; this gives more stable gradients) and softmax.

\item Multiply each value vector $v_j$ by softmax scores.

\item Sum up the weighted value vectors. This is output of self-attention layer for $i$-th word.
\end{enumerate}
In matrix form, suppose our input is $X \in \R^{n \times d},$ where $n$ is the sequence length and $d$ is the embedding dimension. (So rows of $X$ are the inputs.) The QKV matrices are given by $W_Q, W_K, W_V \in \R^{d \times d_k}.$ Then the QKV vectors
\begin{align*}
    Q &= XW_Q \\
    K &= XW_K \\
    V &= XW_V
\end{align*}
are in $R^{n \times d_k}$
For each attention head, we have
\[ \att(Q,K,V) = \softmax\left(\frac{QK^T}{\sqrt{d_k}}\right) \times V \in \R^{n \times d_k}, \]
i.e., aggregate values weighted by attention.

\subsection{Multi-headed attention}
Do $h$ parallel attention heads, with different weight matrices for each head.
\[ \operatorname{MultiHead}(Q, K, V) = \operatorname{Concat}(\head_1, \ldots, \head_h) \times W_O, \]
where
\[ \head_i = \att(Q_i, K_i, V_i) \]
is output of each attention head and $W_O \in \R^{hd_k \times d}.$ Note that $\operatorname{MultiHead}(Q, K, V) \in \R^{n \times d}.$ 

\subsection{Positional encoding}
Transformers are permutation-invariant, so we add positional encoding to input embeddings.

\subsection{FNN}
Output of attention fed into FNN:
\[ \operatorname{FNN}(x) = \operatorname{ReLU}(xW_1 + b_1)W_2 + b_2. \]

\subsection{Residual connections \& layer normalization}
Each sub-layer (attention or feedforward) is followed by
\[ \layernorm(x + \operatorname{Sublayer}(x)). \]
Given $x \in \R^d,$ the layer norm is 
\[ \layernorm(x) = \frac{x - \mu}{\sqrt{\sigma^2 + \epsilon}} \cdot \gamma + \beta, \]
where
\begin{gather*}
    \mu = \frac{\sum x_i}{d} \text{ (mean over features)}\\
    \sigma^2 = \frac{\sum(x_i - \mu)^2}{d} \text{ (variance over features)}\\
    \gamma,\beta \text{: learned scaling and shifting parameters}\\
    \epsilon \text{: small constant to prevent division by zero}.
\end{gather*}

\section{Original Architecture}
Encoder Layer:
\begin{itemize}
    \item Input Embedding + Positional Encoding
    \item Multi-Head Self-Attention \textrightarrow{} Residual + Layer Norm
    \item Feed-Forward Network \textrightarrow{} Residual + Layer Norm
\end{itemize}
Decoder Layer:
\begin{itemize}
\item Masked Multi-Head Self-Attention (prevent peeking at future tokens)
\item Multi-Head Encoder-Decoder Attention \textrightarrow{} Residual + Layer Norm
\item Feed-Forward Network \textrightarrow{} Residual + Layer Norm
\end{itemize}

\section{Modern Architecture}
\begin{itemize}
    \item Input Embedding
    \item Layer Norm \textrightarrow{} Self-Attention \textrightarrow{} Residual
        \begin{itemize}
            \item Grouped-Query Attention
            \item Rotary Embeddings
        \end{itemize}
    \item Layer Norm \textrightarrow{} FNN \textrightarrow{} Residual
\end{itemize}

\subsection{Grouped-query attention}
\begin{itemize}
    \item Use same $W_K$ and $W_V$ across heads, and each head has its own $W_Q.$
    \item Better: divide heads into groups. Heads in the same group share $W_K$ and $W_V$
\end{itemize}

\subsection{Rotary positional embeddings (RoPE)}
\begin{itemize}
	\item Limitations of absolute positional embedding:
	\begin{itemize}
		\item Limited sequence length
		\item Independence of positional embedding, e.g./ difference between position 1 and 2 is the same as the difference between position 1 and 500
	\end{itemize}
    \item Alternative: \textit{relative positional embeddings} \cite{relposrep}
    \begin{itemize}
        \item Bias for positional offsets: use a bias to represent each possible positional offset
        \item Relative attention becomes:
        \begin{equation*}
            \att(Q,K,V)_i = \sum_{j=1}^{n}\softmax(e_{ij}) \times (x_jW_V + a_{ij}^V),
        \end{equation*}
        where
        \begin{equation*}
            e_{ij} = \frac{QK^T + x_iW_Q(a_{ij}^K)^T}{\sqrt{d_k}}.
        \end{equation*}
        \item Here, $a_{ij} \in \R^{1\times d_a}$ is a vector of relative positional weights, i.e.,
        \begin{gather*}
            a_{ij} = w_{\operatorname{clip}(j-i,k)}\\
            \operatorname{clip}(x,k) = \max(-k, \min(k,x)).
        \end{gather*}
        Calculate $w$ for both keys and values.
        \item Clipping allows scalability (i.e., arbitrarily long sequences)
        \item Limitations: slower; complicates key-value cache usage as each additional token changes the embedding for every other token.
    \end{itemize}
    \item For \textit{RoPE} \cite{rotpos}, the intuition is to rotate each embedding by $m \theta,$ where $m$ is the position of the word in the sequence.
    \item Benefits:
        \begin{itemize}
            \item Scalability: adding new words does not change the embedding of previous words
            \item The dot product of the embeddings of two words does not depend on absolute position.
        \end{itemize}
    \item Mathematically, we first work in $\C^{d_k/2}.$ Let $M_j$ be the rotation matrix by $m \theta_j.$ Then the output of RoPE for the $m$-th word is just
    \begin{equation*}
        Q_m\Theta_m = Q_m \times
        \begin{pmatrix}
        M_1 & & & \\
        & M_2 & & \\
        & & \ddots & \\
        & & & M_{d_k/2}
        \end{pmatrix},
    \end{equation*}
    where $Q_m$ is the $m$-th row of $Q$ (i.e., query vector for $m$-th word).
    \item Do the same for key vector.
\end{itemize}

\section{Other Techniques}
\subsection{Sparse attention}
Instead of global autoregressive self-attention, use local autoregressive self-attention in some layers.

\subsection{Mixture of experts (MOE)}
\begin{itemize}
    \item Instead of a single monolithic feedforward layer in the transformer block, use a set of expert networks, and route the input to only a few of them.
    \item A router (a smaller FNN) calculates which experts to turn on.
    \item Could do model merging by doing a weighted average of experts using weights from the router.
\end{itemize}

\section{Fine-Tuning Techniques}

Full fine-tuning, of course, has the best performance but is computationally expensive. Need parameter-efficient fine-tuning.

\subsection{Adapter-based methods}
\begin{itemize}
    \item Adapters: small, trainable models inserted into pre-trained transformers.
    \item Freeze original model weights.
\end{itemize}

\subsubsection{Low-Rank Adaptation (LoRA) \cite{lora}}
\begin{itemize}
    \item Intuition: only train low rank perturbations of the (selected) weight matrices.
    \item Let $W_0 \in \R^{d \times k}$ be a pre-trained weight matrix.
    \item Update: $W_0 + \Delta W = W_0 + BA,$ where $B \in \R^{d \times r}, A \in \R^{r \times d}, r \ll \min(d,k).$
    \item Advantages:
    \begin{itemize}
        \item Roughly converges to full fine-tuning as $r$ increases.
        \item No additional inference latency: when needing to switch to another downstream task, can recover $W_0$ by subtracting $BA,$ and then we can add a new $B'A'.$
    \end{itemize}
    \item Notes \cite{lora_notes}:
    \begin{itemize}
        \item Optimal placement highly dependent on the dataset and model architecture.
        \item For transformers, applying LoRA exclusively to attention layers provides the most stability and mitigates the risk of divergence.
        \item For MoE, applying LoRA to each expert individually boosts performance, but significantly increases memory usage. Applying to router gives limited success.
        \item Could optimize memory by using same $B$ across different $A.$
    \end{itemize}
\end{itemize}